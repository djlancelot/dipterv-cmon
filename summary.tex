%----------------------------------------------------------------------------
\chapter{User Manual, conclusion and future plans}
%----------------------------------------------------------------------------
\section{User Manual for the Monitoring system}

%TODO write user manual

\section{Conclusion}
During the thesis project the author could see the steps of implementing a test environment to a cyberphisical system. Standard description format for sensor data storage of such systems got known. Databases for storing those data was investigated. The one with the most features and plarform independent implementation was chosen. The standard way to describe semantic description for such data and possible database solutions for such systems were also explained. A semantic database with great number of features and Javascript interface has been chosen. After the basics were introduced an actual use case for such a system was shown which was part of the BUTE ETIK project. The implementation of the monitoring system was based on this project. The used technologies were shared and a description of the monitoring system has been given. These technologies were NodeJS with AngularJS frontend using the previously introduced databases. 

The implementation showed that NodeJS is capable of serving the purpose of a monitoring system of Cyberphisical systems. Its great number of tools made it easy to add new features of the system. It is fast and it has great support for many interfaces. It can run with small overhead and it can be installed on most platforms. AngularJS is a great tool for Single page application, it is mobile friendly and supports most of the browsers. As web based technologies arise, this solution is a great way to write monitoring application. 

\section{Future plans}

%TODO Further steps, how to improve