%----------------------------------------------------------------------------
\chapter*{Introduction}\addcontentsline{toc}{chapter}{Introduction}
%----------------------------------------------------------------------------

Nowadays, most of our devices are connected through the Internet. Our computers, mobile phones, surveillance cameras share their information via the world wide web. Each device has many sensors, such as GPS, gravity, acceleration sensor, imaging devices or just processing powers. These sensors or resource's information are stored individually on each device. 

The world is emerging into a state that information needs to be shared between peers and should be stored in an easy to reach independent location called Cloud. The Cloud is a distributed information store which can be reached through the Internet.
This enables sensor information to be stored and processed for new uses, because the same sensor can be used for different purposes. An outdoor surveillance camera can be used to protect from intruders or to provide weather information. This data can be used for different purposes in various regions. A local measurement can control the local heating system, but a grid of weather stations can be used to provide forecast data.
 
 It is not enough to measure all the data using such sensors. It has to be stored for analytics as it can be a source for predictions. Storing has become easier in the BigData world we live in. There are expectations of the method of storing the data. It should be transparent letting different systems share information with each other. To create such a standardized way of storing sensor measurement data Open Geospatial Consorptium started and maintains the SensorMl format. One of SensorML's couple layers is the SOS (Sensor Observation Service) which provides different ways to reach the observed data. This storage engine makes it available for outer services and inner services (like virtual sensors) to reach the desired data and run analytics and trigger monitoring events on them.

Storing sensor data does not give any semantic knowledge about the system. Such knowledge can be that a wind sensor is also a weather sensor or a camera is a visual sensor. To describe such semantic connections an ontology has to be created. There are many ways to store ontologies. Usually they are described as triples like in the most common RDF format but there are newer formats which has extended capabilities and can describe natively much more things ( such as temporal logic) like OWL format. Storing such ontologies are done using special RDF databases. Such databases can analyze connections faster it is even possible gain new knowledge from predefined conditions using reasoning. This way we can describe such things that if an Anemometer is a kind of wind sensors and wind sensors are weather sensors than without specifying explicitly that the Anemometer is a weather sensor the system already knows the answer. 

If we have such an enormous system it can be hard to maintain it. There can be thousands or even more sensors in a network with different capabilities and efficiency. The operation of the sensors should be monitored in an easy to access method, where each sensor's state and measurements should be easy to reach via a modern user interface. The purpose of this document is to describe such a large system's architecture and provide a tool with such a monitoring task can be done.

Such a system is described in this paper. The chosen format is a dynamic, interactive web page. The page consists of an engine that is capable of showing the measurements and state of the sensors using both the RDF and the SOS databases. The sensor's measurements are displayed on a user friendly page. The sensors can be filtered to show only a group of sensors. A status page for the failures are also included to have a big picture of the state of the system. 

In the next two section the basics of the used technologies are described. First there is a detailed introduction to the SOS server and the SensorML standard. After that the semantic description language and its storage engine is introduced. 
In the third chapter the actual system is shown which the monitoring supports. 
There will be a detailed introduction to the components and the used sensors and some use case for the system. The final chapter describes the monitoring system in details. It shows its structure, the used third party frameworks and a usage manual. In the end the whole work will be summarized. 

