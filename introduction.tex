%----------------------------------------------------------------------------
\chapter*{Introduction}\addcontentsline{toc}{chapter}{Introduction}
%----------------------------------------------------------------------------

Nowadays, most of our devices are connected through the Internet. Our computers, mobile phones, surveillance cameras share their information via the world wide web. Each device has many sensors, such as GPS, gravity, acceleration sensor, imaging devices or just processing powers. These sensors or resource's information are stored individually on each device. 

The world is emerging into a state that information needs to be shared between peers and should be stored in an easy to reach independent location called Cloud. 
This enables sensor information to be stored and processed for new uses, because the same sensor can be used for different purposes. An outdoor surveillance camera can be used to protect from intruders or to provide weather information. This data can be used for different purposes in various regions. A local measurement can control the local heating system, but a grid of weather stations can be used to provide forecast data.
 
 The sensor information has to be stored in a standardized way. The common format for such a way is SensorML. One of SensorML's couple layers is the SOS (Sensor Observation Service) which provides different ways to reach the observed data.

Storing sensor data is not enough the system has to have knowledge of the sensors. This is often stored in ontologies or RDF databases.
The provided data is easy to parse for computers but hard to understand by humans.  The main goal of this document to describe a tool that is capable of integrating the two different standards together and transform it into a human readable, user friendly format. 

The chosen format is a dynamic, interactive web page. The page consists of an engine that is capable of filtering the sensors from the chosen data using the RDF database. A sensors data is displayed on a user friendly page that uses some third party tools to display some elements, such as maps.

The SensorML standard and its implementation is described in details in the next chapter. After that the semantic description language and its storage engine is introduced. In the third chapter the fusion of these two systems are presented by getting into the details of the sample implementation.

